\documentclass[a4paper,10pt]{article}
\usepackage[utf8]{inputenc}
\usepackage{amstext}
\usepackage{listings}
\usepackage{graphicx}
\usepackage{subfigure}
\usepackage[colorinlistoftodos]{todonotes}
\usepackage[T1]{fontenc}
\usepackage[utf8]{inputenc}
\usepackage[font=small,labelfont=bf]{caption}
\usepackage{float}
\usepackage[dutch]{babel}
\usepackage[section]{placeins}

\DeclareCaptionLabelFormat{andtable}{#1~#2  \&  \tablename~\thetable}


%opening
\title{Tijdsynchronisatie}
\author{Patrick van Looy \& Bram Leenders}

\begin{document}

\maketitle

\section{Inleiding}
Kijk! Een referentie! \cite{sundararaman2005clock} Wow! En hij staat ook onderaan!

Om draadloze communicatie mogelijk te maken, is er tijdsynchronisatie nodig. Clocks van nodes lopen standaard vrijwel nooit gelijk. Willen verschillende nodes naar elkaar kunnen versturen, dan moeten ze elkaars berichten ACK-en met een daarin een timestamp. Dit vereist wel dat op de nodes die met elkaar communiceren de tijd gelijk moet lopen.

Tijdsynchronisatie kun je op verschillende manieren implementeren. In dit geval gaan we een verschijnsel nabootsen wat we in de natuur ook kunnen vinden. De voorbeelden die hierbij gehanteerd worden, zijn vuurvliegjes en krekels. Vuurvliegjes willen van nature synchroon knipperen. Dit kunnen wij met Arduinos ook implementeren. Tevens kunnen we ook met geluid gaan werken, zoals krekels dat doen door te tjirpen.

Voordat we onze implementatie kunnen maken, zullen we eerst wat gerelateerd werk moeten bestuderen om ons probleem op te lossen. Hierna zullen we een algoritme gaan schrijven en deze testen met Arduinos.

\section{Probleemstelling}
Dit is een probleem!


\section{Tjirpende Arduino's}
De implementatie beschreven in de vorige secties, is niet afhankelijk van het precieze signaal dat de Arduino's geven. Het is alleen afhankelijk van het moment waarop het signaal uitgezonden en ontvangen wordt, en de tijd hiertussen mag niet exorbitant groot worden of wisselend lang en kort duren.

In plaats van een radiosignaal kunnen ook andere signalen uitgewisseld worden, bijvoorbeeld een geluidssignaal. De implementatie hiervan heeft wel wat meer voeten in de aarde, omdat er erg veel ruis is. Tevens heeft de Arduino niet een standaardimplementatie die "pieken" kan detecteren; er is dus functie voor microfoons die vergelijkbaar is met \texttt{radio.available()}.

\subsection{Analoge signaalverwerking}
Een microfoon alleen levert geen geschikt signaal op. Het signaal is te zacht, bevat teveel ruis en is analoog. Om het door de Arduino te laten verwerken moet het signaal versterkt worden en omgezet worden naar een digitaal signaal. Dit doen we in drie stappen:

\todo[inline]{TODO: Drie stappen uitwerken}

\begin{itemize}
	\item High-pass filter om ruis te filteren
	\item Versterker
	\item Omzetten naar digitaal (ADC)
\end{itemize}

\begin{figure}[ht!]
    \centering
    \includegraphics[width=0.8\textwidth]{high_pass_filter_circuit.png}
    \caption{High-pass filter met versterker en ADC.}
    \label{fig:circuit}
\end{figure}


\subsection{Signaalverwerking}
In figuur~\ref{fig:on_off} is zichtbaar dat het versturen van een geluidssignaal een sterk wisselend inputsignaal geeft.
\begin{figure}[ht!]
    \centering
    \includegraphics[width=0.8\textwidth]{resonance_on_off_commit_ff82f.png}
    \caption{Inkomend signaal bij pulserend signaal.}
    \label{fig:on_off}
\end{figure}

\begin{figure}[ht!]
    \centering
    \includegraphics[width=0.8\textwidth]{resonance_2_frequencies_commit_ff82f.png}
    \caption{Inkomend signaal bij verschillende geluidsfrequencies.}
    \label{fig:resonance}
\end{figure}

% \newpage
% \appendix
% \section{Bijlage 1 - Code}
% \label{sec:code}
% % xxxxxxxxxxxxxxxxxxxxxxxxx Code Snippet STARTS xxxxxxxxxxxxxxxxxxxxxx
\lstset{
  language=C,                     % choose the language of the code
  stepnumber=1,                   % the step between two line-numbers. If it's 1, each line will be numbered
  basicstyle=\footnotesize,
 % numbersep=5pt,                 % how far the line-numbers are from the code
%  backgroundcolor=\color{white}, % choose the background color. You must add \usepackage{color}
  showspaces=false,               % show spaces adding particular underscores
  showstringspaces=false,         % underline spaces within strings
  showtabs=false,                 % show tabs within strings adding particular underscores
  tabsize=4,                      % sets default tabsize to 2 spaces
  captionpos=t,                   % sets the caption-position to top
  breaklines=true,                % sets automatic line breaking
  breakatwhitespace=true,         % sets if automatic breaks should only happen at whitespace
 % title=\lstname,                % show the filename of files included with \lstinputlisting;
 % identifierstyle=\color{identifierColor},
 % caption={Array of Pointers to Strings},
 % frame=lrtb,
 % keywordstyle=\color{purple},         % keyword style
 % commentstyle=\color{blue},           % comment style
 % stringstyle=\color{violet},          % string literal style
 belowcaptionskip = 0.2in,            % Space below caption
 abovecaptionskip = 0.2in             % Space above caption
}
% \lstset{language=C}
\begin{lstlisting}
/* Copyright (C) 2011 J. Coliz <maniacbug@ymail.com> */

#include <SPI.h>
#include "nRF24L01.h"
#include "RF24.h"
#include "printf.h"

// Hardware configuration
RF24 radio(3, 9);
const int role_pin = 7;
const uint64_t pipes[2] = { 0x123456789aLL, 0x987654321bLL };
const int sendValue = 170; // binary; 10101010
const int numberOfPackets = 1000;
const int RESETVAL = 42;

// Role of the Arduino; sender or pong-backer
typedef enum { role_ping_out = 1, role_pong_back } role_e;
const char* role_friendly_name[] = { "invalid", "Ping out", "Pong back"};
const rf24_pa_dbm_e outputPowerLevel[] = {RF24_PA_MAX, RF24_PA_HIGH, RF24_PA_LOW, RF24_PA_MIN};
const rf24_datarate_e datarateLevel[] = {RF24_250KBPS, RF24_1MBPS, RF24_2MBPS};

// The role of the current running sketch
role_e role;
void setup(void) {
    pinMode(role_pin, INPUT);
    digitalWrite(role_pin,HIGH);
    delay(20); // Just to get a solid reading on the role pin

    // read the address pin, establish our role
    if ( ! digitalRead(role_pin) )
        role = role_ping_out;
    else
        role = role_pong_back;

    Serial.begin(57600);
    printf_begin();
    printf("\n\rRF24/examples/pingpair/\n\r");
    printf("ROLE: %s\n\r",role_friendly_name[role]);

    // Setup and configure rf radio
    radio.begin();
    radio.setRetries(0,0);
    radio.setDataRate(datarateLevel[0]);
    radio.setPALevel(outputPowerLevel[0]);
    radio.setChannel(0);
    radio.setPayloadSize(8);

    if ( role == role_ping_out ) {
        radio.openWritingPipe(pipes[0]);
        radio.openReadingPipe(1,pipes[1]);
    } else {
        radio.openWritingPipe(pipes[1]);
        radio.openReadingPipe(1,pipes[0]);
    }

    radio.startListening();
    radio.printDetails();
}


// Number of succesfully received packages; 0 <= success <= rounds
int success = 0;
// Rounds of communication so far; 0 <= rounds <= numberOfPackets
int rounds = 0;
int test = 0; int test2 = 0; int testChannel = 0;

void loop(void) {
    if (role == role_ping_out) {
        rounds++;

        radio.stopListening();
        bool ok = radio.write( &sendValue, sizeof(int) );
        radio.startListening();

        // Wait here until we get a response, or timeout (250ms)
        unsigned long started_waiting_at = millis();
        bool timeout = false;
        while ( ! radio.available() && ! timeout )
            if (millis() - started_waiting_at > 250 )
                timeout = true;

        // Describe the results
        if (!timeout) {
            int receivedValue;
            radio.read( &receivedValue, sizeof(int) );

            // Successfull round trip of our value! Increase our success counter.
            if(receivedValue == sendValue) {
                success++;
            }
        }

        if (rounds == numberOfPackets) {
            printf("\n--------\n");
           // printf("Power level: %i (0=MAX, 3=MIN)\n", test);
           // printf("Data rate: %i (0=250kbps, 1=1mbps 2=2mbps)\n", test);
            printf("Channel: %i\n", testChannel);
            printf("# packets sent:               %i\n", numberOfPackets);
            printf("# packets correctly received: %i\n", success);
            printf("--------\n");
            // Reset counters
            success = 0;
            rounds = 0;

            radio.stopListening();
            radio.setPALevel(outputPowerLevel[0]); // Max power; increase chance of succesfully receiving it
            bool ok = radio.write( &RESETVAL, sizeof(int) ); // Pray this will be received
            radio.startListening();

            //test = (test+1)%3;
            test2 = (test2+1)%8; 
            //radio.setDataRate(datarateLevel[test]);
            // radio.setPALevel(outputPowerLevel[test]);
            
            testChannel = (15*test2);
            radio.setChannel(testChannel);
        }

        delay(10);
    }

    // Pong back role.  Receive each packet and send it back
    if ( role == role_pong_back ) {
        if ( radio.available() ) {
            int v;
            bool done = false;
            while (!done) {
                // Read the sent value
                done = radio.read( &v, sizeof(int) );
                delay(10);
            }

            if (v == RESETVAL) {
                //test = (test+1)%3;
                test2 = (test2+1)%8;
                //radio.setDataRate(datarateLevel[test]);
                // radio.setPALevel(outputPowerLevel[test]);
                radio.setChannel(15*test2);
                printf("Finished test; moving to next!\n");
            }

            radio.stopListening();
            radio.write( &v, sizeof(int) );
            radio.startListening();
        }
    }
}
\end{lstlisting}

\bibliographystyle{plain}
\bibliography{verslag_week_4}

\end{document}
