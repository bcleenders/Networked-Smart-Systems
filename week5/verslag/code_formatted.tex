\definecolor{listinggray}{gray}{0.9}
\definecolor{lbcolor}{rgb}{0.9,0.9,0.9}
\lstset{
backgroundcolor=\color{lbcolor},
    tabsize=4,    
%   rulecolor=,
    language=[GNU]C++,
        basicstyle=\scriptsize,
        upquote=true,
        aboveskip={1.5\baselineskip},
        columns=fixed,
        showstringspaces=false,
        extendedchars=false,
        breaklines=true,
        prebreak = \raisebox{0ex}[0ex][0ex]{\ensuremath{\hookleftarrow}},
        frame=single,
        numbers=left,
        showtabs=false,
        showspaces=false,
        showstringspaces=false,
        identifierstyle=\ttfamily,
        keywordstyle=\color[rgb]{0,0,1},
        commentstyle=\color[rgb]{0.026,0.112,0.095},
        stringstyle=\color[rgb]{0.627,0.126,0.941},
        numberstyle=\color[rgb]{0.205, 0.142, 0.73},
%        \lstdefinestyle{C++}{language=C++,style=numbers}’.
}
\lstset{
    backgroundcolor=\color{lbcolor},
    tabsize=4,
  language=C++,
  captionpos=b,
  tabsize=3,
  frame=lines,
  numbers=left,
  numberstyle=\tiny,
  numbersep=5pt,
  breaklines=true,
  showstringspaces=false,
  basicstyle=\footnotesize,
%  identifierstyle=\color{magenta},
  keywordstyle=\color[rgb]{0,0,1},
  commentstyle=\color{teal},
  stringstyle=\color{red}
  }
\begin{lstlisting}
/*
    Positioning system for Arduino One with RF24 radio chip
*/
#include <SPI.h>
#include "nRF24L01.h"
#include "RF24.h"
#include "printf.h"
#include "MatrixMath.h"

#define N (3)
// Kunstmatige waarde voor Z coordinaten
#define Z 1.0
// Percentage verschil (0 < MAX_DIFF <= 1) dat tussen twee metingen mag zitten.
#define MAX_DIFF (0.25)
// Percentage dat de nieuwste meting in het gemiddelde meetelt (0 < WEIGHT <= 1)
// Bij WEIGHT=1 wordt er geen gemiddelde bijgehouden, maar is de nieuwste meting de enige die meetelt.
#define WEIGHT (0.2)

RF24 radio(3, 9);
unsigned long radiotime;
unsigned long audiotime;
unsigned long timelimit = 50000LL;
uint8_t activeBeacon;

float pos[4][2] = { // Positions van de beacons; pos[1][1] is de y positie van beacon 1
    {0.0, 75.0},
    {72.0, 0.0},
    {294.0, 0.0},
    {372.0, 136.0}
};

float D[4];

void setup() {
  // initialize the serial communication:
  Serial.begin(9600);
  printf_begin();

  // Setup and configure rf radio
  radio.begin();
  radio.setRetries(0,0);

  radio.setDataRate(RF24_2MBPS);
  radio.setChannel(76);
  radio.setPayloadSize(1);
  radio.openReadingPipe(1, 0xdeadbeefa1LL);
  radio.openWritingPipe(0xdeadbeefa1LL);
  radio.startListening();
  radio.setAutoAck(false);
}

void loop() {
  while(radio.available()) { 
    radio.read(&activeBeacon, sizeof(uint8_t)); 
  }
  while (! radio.available());

  radiotime = micros();
  radio.read( &activeBeacon, sizeof(uint8_t));

  if(activeBeacon > 3) { return; }

  while(analogRead(A0) < 50) {
    audiotime = micros();
    if(audiotime - radiotime > timelimit) {
      return; 
    }
  }
  
  float diff = audiotime - radiotime;
  diff = diff * 0.03432; // Afstand tot beacon in cm

  //Zwak uitschieters een beetje af: max 30% increase
  if(diff > (D[activeBeacon]* (1.0 + MAX_DIFF)) && D[activeBeacon] > 0) {
    diff = D[activeBeacon] * (1.0 + MAX_DIFF);
  }
  
  if(diff < (D[activeBeacon]* (1.0 - MAX_DIFF))) {
    diff = D[activeBeacon]*(1.0 - MAX_DIFF);
  }
  
  D[activeBeacon] = D[activeBeacon]*(1.0 - WEIGHT) + diff*WEIGHT; // Weer schuivend gemiddelde */

  if(activeBeacon == 3) {
    calcPosition();
  }
}

float A[N][N];
float B[N];

void calcPosition() {
    // Relatieve afstanden tussen de nodes; gebruikt node 3 nog niet!
    A[0][0] = 2*pos[1][0] - 2*pos[0][0]; A[0][1] = 2*pos[1][1] - 2*pos[0][1]; A[0][2] = Z;
    A[1][0] = 2*pos[2][0] - 2*pos[1][0]; A[1][1] = 2*pos[2][1] - 2*pos[1][1]; A[1][2] = Z;
    A[2][0] = 2*pos[0][0] - 2*pos[2][0]; A[2][1] = 2*pos[0][1] - 2*pos[2][1]; A[2][2] = Z;
    Matrix.Invert((float*)A,N);
    B[0] = (D[0]*D[0]) - (D[1]*D[1]) - (pos[0][0]*pos[0][0]) + (pos[1][0]*pos[1][0]) - (pos[0][1]*pos[0][1]) + (pos[1][1]*pos[1][1]);
    B[1] = (D[1]*D[1]) - (D[2]*D[2]) - (pos[1][0]*pos[1][0]) + (pos[2][0]*pos[2][0]) - (pos[1][1]*pos[1][1]) + (pos[2][1]*pos[2][1]);
    B[2] = (D[2]*D[2]) - (D[0]*D[0]) - (pos[2][0]*pos[2][0]) + (pos[0][0]*pos[0][0]) - (pos[2][1]*pos[2][1]) + (pos[0][1]*pos[0][1]);
    float P3[N];
    Matrix.Multiply((float*)A,(float*)B,N,N,1,(float*)P3);    

    // Relatieve afstanden tussen de nodes; gebruikt node 2 nog niet!
    A[0][0] = 2*pos[1][0] - 2*pos[0][0]; A[0][1] = 2*pos[1][1] - 2*pos[0][1]; A[0][2] = Z;
    A[1][0] = 2*pos[3][0] - 2*pos[1][0]; A[1][1] = 2*pos[3][1] - 2*pos[1][1]; A[1][2] = Z;
    A[2][0] = 2*pos[0][0] - 2*pos[3][0]; A[2][1] = 2*pos[0][1] - 2*pos[3][1]; A[2][2] = Z;
    Matrix.Invert((float*)A,N);
    B[0] = (D[0]*D[0]) - (D[1]*D[1]) - (pos[0][0]*pos[0][0]) + (pos[1][0]*pos[1][0]) - (pos[0][1]*pos[0][1]) + (pos[1][1]*pos[1][1]);
    B[1] = (D[1]*D[1]) - (D[3]*D[3]) - (pos[1][0]*pos[1][0]) + (pos[3][0]*pos[3][0]) - (pos[1][1]*pos[1][1]) + (pos[3][1]*pos[3][1]);
    B[2] = (D[3]*D[3]) - (D[0]*D[0]) - (pos[3][0]*pos[3][0]) + (pos[0][0]*pos[0][0]) - (pos[3][1]*pos[3][1]) + (pos[0][1]*pos[0][1]);
    float P2[N];
    Matrix.Multiply((float*)A,(float*)B,N,N,1,(float*)P2);    
    
    // Relatieve afstanden tussen de nodes; gebruikt node 1 nog niet!
    A[0][0] = 2*pos[2][0] - 2*pos[0][0]; A[0][1] = 2*pos[2][1] - 2*pos[0][1]; A[0][2] = Z;
    A[1][0] = 2*pos[3][0] - 2*pos[2][0]; A[1][1] = 2*pos[3][1] - 2*pos[2][1]; A[1][2] = Z;
    A[2][0] = 2*pos[0][0] - 2*pos[3][0]; A[2][1] = 2*pos[0][1] - 2*pos[3][1]; A[2][2] = Z;
    Matrix.Invert((float*)A,N);
    B[0] = (D[0]*D[0]) - (D[2]*D[2]) - (pos[0][0]*pos[0][0]) + (pos[2][0]*pos[2][0]) - (pos[0][1]*pos[0][1]) + (pos[2][1]*pos[2][1]);
    B[1] = (D[2]*D[2]) - (D[3]*D[3]) - (pos[2][0]*pos[2][0]) + (pos[3][0]*pos[3][0]) - (pos[2][1]*pos[2][1]) + (pos[3][1]*pos[3][1]);
    B[2] = (D[3]*D[3]) - (D[0]*D[0]) - (pos[3][0]*pos[3][0]) + (pos[0][0]*pos[0][0]) - (pos[3][1]*pos[3][1]) + (pos[0][1]*pos[0][1]);
    float P1[N];
    Matrix.Multiply((float*)A,(float*)B,N,N,1,(float*)P1);    
    
    // Relatieve afstanden tussen de nodes; gebruikt node 0 nog niet!
    A[0][0] = 2*pos[2][0] - 2*pos[1][0]; A[0][1] = 2*pos[2][1] - 2*pos[1][1]; A[0][2] = Z;
    A[1][0] = 2*pos[3][0] - 2*pos[2][0]; A[1][1] = 2*pos[3][1] - 2*pos[2][1]; A[1][2] = Z;
    A[2][0] = 2*pos[1][0] - 2*pos[3][0]; A[2][1] = 2*pos[1][1] - 2*pos[3][1]; A[2][2] = Z;
    Matrix.Invert((float*)A,N);
    B[0] = (D[1]*D[1]) - (D[2]*D[2]) - (pos[1][0]*pos[1][0]) + (pos[2][0]*pos[2][0]) - (pos[1][1]*pos[1][1]) + (pos[2][1]*pos[2][1]);
    B[1] = (D[2]*D[2]) - (D[3]*D[3]) - (pos[2][0]*pos[2][0]) + (pos[3][0]*pos[3][0]) - (pos[2][1]*pos[2][1]) + (pos[3][1]*pos[3][1]);
    B[2] = (D[3]*D[3]) - (D[1]*D[1]) - (pos[3][0]*pos[3][0]) + (pos[1][0]*pos[1][0]) - (pos[3][1]*pos[3][1]) + (pos[1][1]*pos[1][1]);
    float P0[N];
    Matrix.Multiply((float*)A,(float*)B,N,N,1,(float*)P0);
    
    // Bereken het gemiddelde van de verschillende metingen:
    int avg[N];
    avg[0] = (int) (P0[0] + P1[0] + P2[0] + P3[0]) / 4.0;
    avg[1] = (int) (P0[1] + P1[1] + P2[1] + P3[1]) / 4.0;
    avg[2] = (int) (P0[2] + P1[2] + P2[2] + P3[2]) / 4.0;
  
    printf("Position: (%d,%d)\n\n", avg[0], avg[1]);
}
\end{lstlisting}