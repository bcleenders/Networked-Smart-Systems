
\documentclass[a4paper,10pt]{article}
\usepackage[utf8]{inputenc}
\usepackage{amstext}
\usepackage{listings}
\usepackage{graphicx}
\usepackage{subfigure}
\usepackage[colorinlistoftodos]{todonotes}
\usepackage[T1]{fontenc}
\usepackage[utf8]{inputenc}
\usepackage[font=small,labelfont=bf]{caption}
\usepackage{float}
\usepackage[dutch]{babel}
\usepackage[section]{placeins}
\usepackage{algorithm}
\usepackage{algpseudocode}
\usepackage{algorithmicx}

\DeclareCaptionLabelFormat{andtable}{#1~#2  \&  \tablename~\thetable}


%opening
\title{Indoor positioning met Arduino's}
\author{Bram Leenders \& Patrick van Looy}

\begin{document}

\maketitle

\section{Inleiding}
...

\section{Gerelateerd werk}\label{sec:gerelateerd}
TOF (Time of flight) vs. RSSI (received signal strength indication) vs. TDOA (time difference of arrival)
Er zijn verschillende manier om met behulp van radio en/of geluidssignalen een afstand te meten, we zullen de volgende drie kort toelichten:
\begin{itemize}
    \item Received Signal Strength Indication (RSSI)
    \item Time Difference of Arrival (TDOA)
    \item Time of Flight (TOF)
\end{itemize}

\subsection{Time of Flight}
Voor het onderzoek van deze paper is de time of flight (TOF) gebruikt om de afstand tussen beacons en de ontvanger te meten. Bij deze methode wordt gebruik gemaakt van het verschil in propagatiesnelheid van licht en geluid; radiosignalen gaan met lichtsnelheid (ca. $3\cdot 10^{8}$m/s), maar geluid gaat veel langzamer (ca $340$ m/s). Door beacons tegelijkertijd een radio- en een geluidssignaal uit te laten zenden kan met behulp van het verschil in ontvangsttijden de afstand tussen het beacon en een ontvanger berekend worden. Deze techniek wordt beschreven door Barshan en Ballur~\cite{barshan2000fast}.

Figuur~\ref{fig:tijdsdiagram} geeft een voorbeeld van inkomende signalen bij een dergelijke aanpak.
\begin{figure}[ht!]
    \centering
    \includegraphics[width=0.7\textwidth]{tijdsdiagram.png}
    \caption{Tijdsdiagram radio en microfoon input. \textit{Bron: \cite{park2011beacon}}}
    \label{fig:tijdsdiagram}
\end{figure}

\section{Implementatie}
De implementatie gebruikt de time of flight (TOF) om de afstand tussen beacons en de ontvanger te meten.




\section{Resultaten en discussie}\label{sec:resultaten}
...

\section{Conclusie}\label{sec:conclusie}
...

\newpage
\appendix
\section{Bijlage 1 - Code}
\label{sec:code}
% xxxxxxxxxxxxxxxxxxxxxxxxx Code Snippet STARTS xxxxxxxxxxxxxxxxxxxxxx
\lstset{
  language=C,                     % choose the language of the code
  stepnumber=1,                   % the step between two line-numbers. If it's 1, each line will be numbered
  basicstyle=\footnotesize,
 % numbersep=5pt,                 % how far the line-numbers are from the code
%  backgroundcolor=\color{white}, % choose the background color. You must add \usepackage{color}
  showspaces=false,               % show spaces adding particular underscores
  showstringspaces=false,         % underline spaces within strings
  showtabs=false,                 % show tabs within strings adding particular underscores
  tabsize=4,                      % sets default tabsize to 2 spaces
  captionpos=t,                   % sets the caption-position to top
  breaklines=true,                % sets automatic line breaking
  breakatwhitespace=true,         % sets if automatic breaks should only happen at whitespace
 % title=\lstname,                % show the filename of files included with \lstinputlisting;
 % identifierstyle=\color{identifierColor},
 % caption={Array of Pointers to Strings},
 % frame=lrtb,
 % keywordstyle=\color{purple},         % keyword style
 % commentstyle=\color{blue},           % comment style
 % stringstyle=\color{violet},          % string literal style
 belowcaptionskip = 0.2in,            % Space below caption
 abovecaptionskip = 0.2in             % Space above caption
}
% \lstset{language=C}
\begin{lstlisting}
/* Copyright (C) 2011 J. Coliz <maniacbug@ymail.com> */

#include <SPI.h>
#include "nRF24L01.h"
#include "RF24.h"
#include "printf.h"

// Hardware configuration
RF24 radio(3, 9);
const int role_pin = 7;
const uint64_t pipes[2] = { 0x123456789aLL, 0x987654321bLL };
const int sendValue = 170; // binary; 10101010
const int numberOfPackets = 1000;
const int RESETVAL = 42;

// Role of the Arduino; sender or pong-backer
typedef enum { role_ping_out = 1, role_pong_back } role_e;
const char* role_friendly_name[] = { "invalid", "Ping out", "Pong back"};
const rf24_pa_dbm_e outputPowerLevel[] = {RF24_PA_MAX, RF24_PA_HIGH, RF24_PA_LOW, RF24_PA_MIN};
const rf24_datarate_e datarateLevel[] = {RF24_250KBPS, RF24_1MBPS, RF24_2MBPS};

// The role of the current running sketch
role_e role;
void setup(void) {
    pinMode(role_pin, INPUT);
    digitalWrite(role_pin,HIGH);
    delay(20); // Just to get a solid reading on the role pin

    // read the address pin, establish our role
    if ( ! digitalRead(role_pin) )
        role = role_ping_out;
    else
        role = role_pong_back;

    Serial.begin(57600);
    printf_begin();
    printf("\n\rRF24/examples/pingpair/\n\r");
    printf("ROLE: %s\n\r",role_friendly_name[role]);

    // Setup and configure rf radio
    radio.begin();
    radio.setRetries(0,0);
    radio.setDataRate(datarateLevel[0]);
    radio.setPALevel(outputPowerLevel[0]);
    radio.setChannel(0);
    radio.setPayloadSize(8);

    if ( role == role_ping_out ) {
        radio.openWritingPipe(pipes[0]);
        radio.openReadingPipe(1,pipes[1]);
    } else {
        radio.openWritingPipe(pipes[1]);
        radio.openReadingPipe(1,pipes[0]);
    }

    radio.startListening();
    radio.printDetails();
}


// Number of succesfully received packages; 0 <= success <= rounds
int success = 0;
// Rounds of communication so far; 0 <= rounds <= numberOfPackets
int rounds = 0;
int test = 0; int test2 = 0; int testChannel = 0;

void loop(void) {
    if (role == role_ping_out) {
        rounds++;

        radio.stopListening();
        bool ok = radio.write( &sendValue, sizeof(int) );
        radio.startListening();

        // Wait here until we get a response, or timeout (250ms)
        unsigned long started_waiting_at = millis();
        bool timeout = false;
        while ( ! radio.available() && ! timeout )
            if (millis() - started_waiting_at > 250 )
                timeout = true;

        // Describe the results
        if (!timeout) {
            int receivedValue;
            radio.read( &receivedValue, sizeof(int) );

            // Successfull round trip of our value! Increase our success counter.
            if(receivedValue == sendValue) {
                success++;
            }
        }

        if (rounds == numberOfPackets) {
            printf("\n--------\n");
           // printf("Power level: %i (0=MAX, 3=MIN)\n", test);
           // printf("Data rate: %i (0=250kbps, 1=1mbps 2=2mbps)\n", test);
            printf("Channel: %i\n", testChannel);
            printf("# packets sent:               %i\n", numberOfPackets);
            printf("# packets correctly received: %i\n", success);
            printf("--------\n");
            // Reset counters
            success = 0;
            rounds = 0;

            radio.stopListening();
            radio.setPALevel(outputPowerLevel[0]); // Max power; increase chance of succesfully receiving it
            bool ok = radio.write( &RESETVAL, sizeof(int) ); // Pray this will be received
            radio.startListening();

            //test = (test+1)%3;
            test2 = (test2+1)%8; 
            //radio.setDataRate(datarateLevel[test]);
            // radio.setPALevel(outputPowerLevel[test]);
            
            testChannel = (15*test2);
            radio.setChannel(testChannel);
        }

        delay(10);
    }

    // Pong back role.  Receive each packet and send it back
    if ( role == role_pong_back ) {
        if ( radio.available() ) {
            int v;
            bool done = false;
            while (!done) {
                // Read the sent value
                done = radio.read( &v, sizeof(int) );
                delay(10);
            }

            if (v == RESETVAL) {
                //test = (test+1)%3;
                test2 = (test2+1)%8;
                //radio.setDataRate(datarateLevel[test]);
                // radio.setPALevel(outputPowerLevel[test]);
                radio.setChannel(15*test2);
                printf("Finished test; moving to next!\n");
            }

            radio.stopListening();
            radio.write( &v, sizeof(int) );
            radio.startListening();
        }
    }
}
\end{lstlisting}

\bibliographystyle{plain}
\bibliography{verslag_week_5}

\end{document}
