% xxxxxxxxxxxxxxxxxxxxxxxxx Code Snippet STARTS xxxxxxxxxxxxxxxxxxxxxx
\lstset{
  language=C,                     % choose the language of the code
  stepnumber=1,                   % the step between two line-numbers. If it's 1, each line will be numbered
  basicstyle=\footnotesize,
 % numbersep=5pt,                 % how far the line-numbers are from the code
%  backgroundcolor=\color{white}, % choose the background color. You must add \usepackage{color}
  showspaces=false,               % show spaces adding particular underscores
  showstringspaces=false,         % underline spaces within strings
  showtabs=false,                 % show tabs within strings adding particular underscores
  tabsize=4,                      % sets default tabsize to 2 spaces
  captionpos=t,                   % sets the caption-position to top
  breaklines=true,                % sets automatic line breaking
  breakatwhitespace=true,         % sets if automatic breaks should only happen at whitespace
 % title=\lstname,                % show the filename of files included with \lstinputlisting;
 % identifierstyle=\color{identifierColor},
 % caption={Array of Pointers to Strings},
 % frame=lrtb,
 % keywordstyle=\color{purple},         % keyword style
 % commentstyle=\color{blue},           % comment style
 % stringstyle=\color{violet},          % string literal style
 belowcaptionskip = 0.2in,            % Space below caption
 abovecaptionskip = 0.2in             % Space above caption
}
% \lstset{language=C}
\begin{lstlisting}
/* Copyright (C) 2011 J. Coliz <maniacbug@ymail.com> */

#include <SPI.h>
#include "nRF24L01.h"
#include "RF24.h"
#include "printf.h"

RF24 radio(3, 9);

// sets the role of this unit in hardware.  Connect to GND to be the 'pong' receiver
// Leave open to be the 'ping' transmitter
const int role_pin_sender = 7;
const int role_repeat = 6;

// Radio pipe addresses for the 3 nodes to communicate.
const uint64_t pipes[3] = { 0x123456789aLL, 0x987654321bLL, 0x384654f2cbLL };

const int sendValue = 170; // binary; 10101010
const int numberOfPackets = 1000;
const int RESETVAL = 42;

typedef enum { role_sender = 1, role_receiver = 2, role_repeater = 3 } role_e;
// The debug-friendly names of those roles
const char* role_friendly_name[] = { "invalid", "Sender", "Receiver", "Repeater"};

// The role of the current running sketch
role_e role;

void setup(void) {
    // set up the role pin
    pinMode(role_pin_sender, INPUT);
    digitalWrite(role_pin_sender,HIGH);
    delay(20); // Just to get a solid reading on the role pin
    // read the address pin, establish our role
    if (!digitalRead(role_pin_sender)) {
        role = role_sender; // sender        
    }
    else {
        // set up the role pin
        pinMode(role_repeat, INPUT);
        digitalWrite(role_repeat,HIGH);
        delay(20); // Just to get a solid reading on the role pin

        if (!digitalRead(role_repeat)) {
            role = role_repeater; // repeater
        }
        else {
            role = role_receiver; // receiver
        }
    }

    Serial.begin(57600);
    printf_begin();

    radio.begin();
    radio.setRetries(0,0);

    radio.setDataRate(RF24_250KBPS);
    radio.setPALevel(RF24_PA_MAX);
    radio.setChannel(0);

    // optionally, reduce the payload size.  seems to
    // improve reliability
    radio.setPayloadSize(8);

    if ( role == role_sender ) { // Sender
        radio.openWritingPipe(pipes[1]); // Write to repeater
        radio.openReadingPipe(1,pipes[0]); // Read from repeater
    }
    else if (role == role_repeater){ // Repeater
        radio.openReadingPipe(1,pipes[1]);
    }
    else { // Receiver
        radio.openWritingPipe(pipes[1]);
        radio.openReadingPipe(1,pipes[2]);
    }

    radio.startListening();
    radio.printDetails();
}

// Bevat het nummer dat nu gestuurd moet worden
int currentNumber = 1;

void resetRadioReads(void) {
    radio.stopListening();
    radio.openWritingPipe(0x111111110aLL);
}

void loop(void) {
    if (role == role_sender) {
        radio.stopListening();
        radio.openWritingPipe(pipes[1]);
        bool ok = radio.write( &currentNumber, sizeof(int) );
        resetRadioReads();
        radio.startListening();

        // Wait here until we get a response, or timeout (250ms)
        unsigned long started_waiting_at = millis();
        bool timeout = false;
        while ( ! radio.available() && ! timeout )
            if (millis() - started_waiting_at > 250 )
                timeout = true;

        // Describe the results
        if ( timeout ) {
          // Zend opnieuw!
          // currentNumber wordt opnieuw verzonden in de volgende iteratie van loop()
          printf("Timeout occurred at sender; no ACK received. Packet #: %i\n", currentNumber);
        }
        else {
            int receivedValue;
            radio.read( &receivedValue, sizeof(int) );

            // ACK our value! Increase our success counter.
            if(receivedValue == - currentNumber) {
                // Success!
                printf("ACK succesfully received for packet #%i\n", currentNumber);
                currentNumber++;
            }
            else {
              // Received double ACK
              printf("Received double ACK. Packet #: %i\n", currentNumber);
            }
        }
        delay(50);
    }

    if ( role == role_receiver ) {
        // if there is data ready
        if ( radio.available() ) {
            
            // Dump the payloads until we've gotten everything
            int v; bool done = false;
            while (!done) {
                done = radio.read( &v, sizeof(int) );
                delay(10);
            }

            printf("Received packet # %i\n", v);

            v = -v; // ACK waarde is negatief

            radio.stopListening();
            radio.openWritingPipe(pipes[1]);
            radio.write( &v, sizeof(int) );
            resetRadioReads();
            radio.startListening();
        }
    }
    if (role==role_repeater) {
        if ( radio.available() ) {
            int v; bool done = false;
            while (!done) {
                done = radio.read( &v, sizeof(int) );
                delay(10);
            }
            radio.stopListening();

            if (v > 0) { // Dit is het originele bericht; stuur naar receiver
                radio.openWritingPipe(pipes[2]);
            }
            else { // Dit is de ACK; stuur naar sender
                radio.openWritingPipe(pipes[0]);
            }

            radio.write( &v, sizeof(int) );
            resetRadioReads();
            radio.startListening();
        }
    }
}
\end{lstlisting}